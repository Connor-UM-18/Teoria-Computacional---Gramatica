\chapter{Introducción}
La práctica consiste en elaborar un programa para procesar una gramática no ambigua. En esta gramática, se definen dos reglas de producción:\newline
\begin{enumerate}
    \item B -> (RB | ε: Esta regla indica que el símbolo no terminal B puede ser reemplazado por la secuencia $'$(RB$'$ o puede ser eliminado, es decir, puede ser el símbolo vacío.\newline
    \item R -> ) | (RR: Esta regla indica que el símbolo no terminal R puede ser reemplazado por el símbolo $'$)$'$ o puede ser seguido por la secuencia $'$(RR$'$.\newline
\end{enumerate}

El objetivo del programa es tomar una cadena de entrada conformada por paréntesis balanceados y aplicar una derivación única y a la izquierda para obtener una cadena generada. El proceso de derivación se realiza escaneando la cadena de izquierda a derecha y aplicando las reglas de producción correspondientes.\newline

El programa debe contar con las siguientes características adicionales:\newline
\begin{enumerate}
    \item La cadena de entrada puede ser ingresada por el usuario o generada automáticamente.\newline
    \item La evaluación de la gramática debe ser impresa en pantalla y/o guardada en un archivo. Se debe indicar el símbolo que se está evaluando, la producción aplicada y la cadena generada en cada paso.\newline
    \item El programa debe ser capaz de manejar cadenas de hasta 1,000 caracteres de longitud.\newline
\end{enumerate}

Es importante seguir las condiciones de evaluación establecidas por el docente, como subir un informe en formato PDF al aula virtual, adjuntando el código fuente en LaTeX.\newline

Esta práctica se basa en conceptos de teoría de la computación, específicamente en el estudio de gramáticas formales y sus derivaciones. El propósito es aplicar estos conceptos para implementar un programa que pueda procesar y generar cadenas válidas según una gramática específica.\newline
