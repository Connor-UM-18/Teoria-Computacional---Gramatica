\chapter{Anexos}
\lstset{
    language=C,
    basicstyle=\ttfamily\small\color{black},
    numbers=left,
    numberstyle=\tiny,
    stepnumber=1,
    numbersep=8pt,
    backgroundcolor=\color{white},
    showspaces=false,
    showstringspaces=false,
    showtabs=false,
    frame=single,
    rulecolor=\color{magenta},
    tabsize=2,
    captionpos=b,
    breaklines=true,
    breakatwhitespace=false,
    title=\lstname,
    escapeinside={\%*}{*)},
    keywordstyle=\color{blue},
    commentstyle=\color{red},
    stringstyle=\color{orange},
    morecomment=[l][\color{red}]{\#},
    otherkeywords={=,!,<,>,*,+,-,&,|,^,~},
    numbers=left,                   % Coloca los números de línea a la izquierda
    numberstyle=\tiny\color{black}, % Estilo de los números de línea
    stepnumber=1,    % Incremento en el que se muestran los números de línea
    numbersep=8pt
}
\section{Código LATEX de este documento}
Dirección GitHub:\url{}\newline
Dirección Overleaf:\url{https://www.overleaf.com/7893931279kkfrjsrcvgzx}\newline
\section{gramatica.py}
Se presenta el código implementado para la solución al problema con extensión .py . Donde es necesario tener importada la libreria random. \newline
\\
\begin{lstlisting}
#Teoria de la computacion
#Gramatica
#Alumno: Connor Urbano Mendoza

import random
#()()()()()()()()()()()()()(()()()()()()()()()()()()()()(()()()()()()()()()()()()()()()()))

# Función para generar automáticamente una cadena de paréntesis balanceados
def generate_balanced_parentheses(n):
    parentheses = ['(', ')']
    return ''.join(random.choice(parentheses) for _ in range(n))

def process_grammar(input_string):
    steps = "B"
    output = []
    i = 0
    posicion=0
    bandera=0
    contador=1
    with open('C:\\Users\\soyco\\OneDrive\\Documents\\ESCOM\\sem4\\Teoria\\P2\\gramatica\\output\\evaluacion.txt', 'w') as archivo:
        archivo.write("Evaluacion de la gramatica:"+ '\n')
        if i == len(input_string):
            return 'Invalid input string.'
        while 1:
            
            archivo.write("-------------------------------------"+ '\nPaso : '+str(contador)+'\n')
            if i >= 0 and i < len(input_string) and bandera==0:
                substring = input_string[i:]
                archivo.write("Input en curso: "+substring+ "     "+steps+". <--- Pasos de la derivacion mas a la izquierda.\n")
                archivo.write("Simbolo siguiente: "+input_string[i]+ '.\n')
            elif input_string=='ε':
                archivo.write("Input en curso:  E     "+steps+". <--- Pasos de la derivacion mas a la izquierda.\n")
                archivo.write("Simbolo siguiente:  E.\n")
            

            for symbol in steps:
                if symbol=="B":
                    LeftMostDerivation = "B"
                    break
                elif symbol=="R":
                    LeftMostDerivation = "R"
                    break
                posicion = posicion+1
            
            if LeftMostDerivation == 'B' and input_string[i] == '(':
                index = posicion
                steps = steps[:index] + '(RB' + steps[index+1:]  
                archivo.write("Regla aplicada: B --> (RB \n")
            elif LeftMostDerivation == 'B' and input_string[i] == '':
                index = posicion
                steps = steps[:index] + '' + steps[index+1:]  
                archivo.write("Regla aplicada: B --> E \n")
            elif LeftMostDerivation == 'R' and input_string[i] == ')':
                index = posicion
                steps = steps[:index] + ')' + steps[index+1:]  
                archivo.write("Regla aplicada: R --> ) \n")
            elif LeftMostDerivation == 'R' and input_string[i] == '(':
                index = posicion
                steps = steps[:index] + '(RR' + steps[index+1:]  
                archivo.write("Regla aplicada: R --> (RR \n")
            elif LeftMostDerivation == 'B' and input_string=='ε':
                index = posicion
                steps = steps[:index] + '' + steps[index+1:]
                archivo.write("Regla aplicada: B --> E \n")
                archivo.write("\n-------------------------------------"+ '\nPaso : '+str(contador+1)+'\n')
                archivo.write(steps+"  <--- Evaluacion resultante.\n")
                archivo.write("Exitosa. Cadena valida\n")
                return steps
            elif LeftMostDerivation == 'R' and input_string=='ε':
                archivo.write("Regla aplicada: R --> E \n")
                archivo.write("\n-------------------------------------"+ '\nPaso : '+str(contador+1)+'\n')
                archivo.write(steps+"  <--- Evaluacion resultante.\n")
                archivo.write("Error: Fuera de condicionales. Cadena invalida \n")
                return steps
            posicion=0 
            i =i+1
            contador=contador+1
            if i == len(input_string) or bandera==1:
                bandera=1
                i=0
                input_string='ε'



# Main
input_option = input("¿Deseas ingresar una cadena manualmente (1)\n o generarla automáticamente (2)?\n ")

if input_option == '1':
    input_string = input("Ingresa una cadena de paréntesis balanceados: ")
elif input_option == '2':
    numero=random.randint(1, 1000)
    print("Tamanio de la cadena escogido aleatoriamente: "+str(numero))
    input_string = generate_balanced_parentheses(numero)
    print("Cadena generada automáticamente:", input_string)
else:
    print("Opción inválida. El programa se cerrará.")
    exit()

derivation = process_grammar(input_string)
print("\n\nDerivación resultante:", derivation)

\end{lstlisting}